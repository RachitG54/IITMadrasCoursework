\documentclass[11pt]{article}

\usepackage{gitinfo}
\usepackage{amsmath,amssymb,amsfonts,amsthm}
\usepackage[linkcolor=blue,colorlinks=true]{hyperref}
\usepackage{color}
\usepackage{fullpage}
\usepackage[svgnames]{xcolor}
\usepackage{caption}
\usepackage{verbatim,graphicx}
\usepackage{complexity}
\usepackage[normalem]{ulem}		% For strike out
\usepackage[ruled,vlined]{algorithm2e}
\usepackage{float}
\usepackage{multicol}
\usepackage{array}

\newcommand{\Latt}{\mathcal{L}}

%Commenting and TODO listing commands

\usepackage{todonotes}
\newcounter{todocounter}
\newcommand{\todonum}[2][]{\stepcounter{todocounter}\todo[#1]{\thetodocounter: #2}}

\theoremstyle{plain}% default
\newtheorem{theorem}{Theorem}[section]
\newtheorem{lemma}[theorem]{Lemma}
\newtheorem{proposition}[theorem]{Proposition}
\newtheorem{claim}[theorem]{Claim}
\newtheorem{corollary}[theorem]{Corollary}
\newtheorem{conjecture}[theorem]{Conjecture}

%\theoremstyle{remark}
\newtheorem{note}[theorem]{Note}
\newtheorem{observation}[theorem]{Observation}
\newtheorem{question}[theorem]{Question}
\newtheorem{remark}[theorem]{Remark}

\theoremstyle{definition}
\newtheorem{definition}[theorem]{Definition}
\newtheorem{exercise}[theorem]{Exercise}
\newtheorem{problem}[theorem]{Problem}

\newcommand{\field}{\mathbb{F}}
\newcommand{\F}{{\mathbb{F}}}
\newcommand{\N}{{\mathbb{N}}}
\renewcommand{\R}{{\mathbb{R}}}

\newcommand{\size}{\textrm{\bf SIZE}}

\newlang{\MCSP}{MCSP}
\newlang{\REACH}{REACH}

\newcommand{\set}[1]{\left\{ #1 \right\}}
\newcommand{\spa}[1]{\textrm{span} \set{#1}}

\title{Efficient Collision-Resistant Hashing}
\author{Rachit Garg \\ \texttt{CS14B050}}
\date{\today}
\begin{document}
\maketitle
\begin{center}
Based on paper\\Efficient Collision-Resistant Hashing from
Worst-Case Assumptions on Cyclic Lattices \\Published by Chris Peikert  and Alon Rosen\\ Work appeared in 3 rd Theory of Cryptography Conference (TCC 2006) \cite{Peikert:2006:ECH:2180286.2180297}.
\end{center}

\tableofcontents
\section{Problem Motivation}
Collision-resistant hash functions
are one of the most widely-employed cryptographic primitives. Their applications include integrity
checking, user and message authentication, commitment protocols, and more.
Many of the applications of collision-resistant hashing tend to invoke the hash function only a
small number of times. Thus, the efficiency of the function has a direct effect on the efficiency of the
application that uses it. Collision-resistance can be obtained from many well-studied complexity assumptions, but the
resulting hash functions are not efficient enough for practical use. Instead, faster heuristic constructions such as MD5 and SHA-1 are often employed. Unfortunately, recent cryptanalytic analysis of many popular hash functions casts doubt on the heuristic approach \cite{Wang:2005:FCF:2153419.2153421},\cite{Wang:2005:BMO:2154598.2154601}. This paper proposed an efficient collision-resistant hash function with rigorous security guarantees.
\section{Context and problem statement}
\textit{Collision resistant functions}: A function family $\{f_a\}, a \in A$ is said to be collision-resistant if given a uniformly chosen $a \in A$, it is infeasible to find elements $x_1 \neq x_2$ so that $f_a(x_1) = f_a(x_2)$.
\\
\textit{Generalized Knapsacks}: For a ring R, key $a = (a_1 , . . . , a_m) \in R^m$, and input $x = (x_1 , . . . , x_m)$.
\begin{center}
$f_a(x) = \sum_{i=1}^{m}a_i·x_i$
\end{center}
where each $x_i$ is restricted to some large subset $S \subseteq R$. This generalization was proposed by Micciancio, who suggested a specific choice of the ring $R$ and subset $S$ for which inverting the function (for random $a$, $x$) is at least as hard as solving certain worst-case problems on cyclic lattices \cite{Micciancio:2002:GCK:645413.652130}.
\\
The authors mention that even though many knapsack systems have been broken heuristically, there is still no asymptotically-efficient attack on the general function. The paper studied continued Micciancio’s line of study, and showed that for a different choice of $S \subset R$, the generalized knapsack function can enjoy even stronger cryptographic properties.
\subsection{Cyclic Lattices}
Lattices admit worst-case to average-case reductions. Ajtai first constructed a one-way function \cite{Ajtai:1996:GHI:237814.237838}, which was later observed to also be collision-resistant \cite{Bellare:1997:NPC:1754542.1754560}. These constructions tended to be asymptotically more efficient than those based on, e.g., modular exponentiation. An interesting special case is presented by cyclic lattices. 
\\
A lattice $L$ is said to be cyclic if for any
vector $x \in L$, its cyclic rotation also belongs to $L$. The cyclic rotation of $x = (x_0 , . . . , x_{n−1})^T \in R^n$ is defined as $(x_{n-1}, x_0, . . . , x_{n-2})^T$.
\\
Currently no hardness results are known for problems on cyclic lattices (even in their exact versions), and the additional structure may indeed reduce the underlying hardness. However, state-of-the-art lattice algorithms appear not to benefit from cyclicity, and it seems reasonable to conjecture that standard problems on cyclic lattices are intractable, at least for small approximation factors.
\section{Comparison with known solutions}
	\begin{table}[H]
		\begin{tabular}{|m{2cm} | m{2.1cm} | m{2.3cm} | m{2.6cm} | m{3.1cm} | m{3cm}|}
			\hline
			\textbf{ } & \textbf{Security} & \textbf{Efficiency} & \textbf{Lattice Class} & \textbf{Assumption} & \textbf{Approx. Factor}\\
			\hline
			Ajtai & CRHF & $O(n^2)$ & General & SVP etc. & poly(n) \\ \hline
			Cai, Nerurkar & CRHF & $O(n^2)$ & General & SVP etc. & $n^{4+\epsilon}$ \\ \hline
			Micciancio \vspace{0.3cm} & OWF & $\tilde{O}(n)$ & Cyclic & GDD & $n^{1+\epsilon}$ \\ \hline
			Micciancio, Regev & CRHF & $O(n^2)$ & General & SVP etc. & $\tilde{O}(n)$ \\ \hline
			This \vspace{0.3cm} work& CRHF & $\tilde{O}(n)$ & Cyclic & SVP etc. & $\tilde{O}(n)$ \\
			\hline
		\end{tabular}
		\caption{Comparison of results in lattice-based cryptographic functions with worst-case to average case security reductions, to date. “Efficiency” means the key size and computation time, as a function of the lattice dimension n. “Security” denotes the function’s main cryptographic property.}
	\end{table}
The work studied was similar to Micciancio's work on cylic lattices. However the reduction used to establish collision-resistance differs in many significant ways. Micciancio’s function is proven to be one-way, while the authors is collision-resistant. On the other hand, Micciancio relies on a presumably weaker worst-case assumption than theirs. The stronger assumption, combined with the algebraic
view of cyclic lattices, makes the security reduction tighter and conceptually simpler.
\par
Other works considered are slower in efficiency. The structure of cyclic lattices and the choice of ring admits very efficient implementations of the knapsack function: using a Fast Fourier Transform algorithm. The resulting time complexity of the function is $O(m \cdot n \cdot poly(\log n))$, with key size $O(m \cdot n \log n)$. Where $m$ denotes the number of elements in the key.
\section{Paper results}
They formulated a reduction showing that for cyclic lattices of prime dimension $n$, the short independent vectors problem $SIVP$ reduces to (a slight variant of) the shortest vector problem $SVP$ with only a factor of $2$ loss in approximation factor. They note that factor of loss in approximation is not trivial and the prime dimension constraint is not restricting. For general lattices, the best known reduction loses a $\sqrt n$ factor \cite{Micciancio2002}; furthermore, that reduction performs manipulations on its input lattice that can destroy the cyclicity property. Hence their reduction can be seen as the first connection between SIVP and SVP on cyclic lattices.
\par
Also in using the Gaussian techniques of \cite{Micciancio:2007:WAR:1328722.1328733}, they also establish a new bound on the discrete Gaussian distribution over general lattices, which may be of independent interest.
\par
Their main result is that certain instantiations of the generalized knapsack function are collision resistant, assuming it is infeasible to approximate the shortest vector in cyclic lattices up to factors $\tilde O(n)$ almost linear in the dimension $n$. The construction is also efficient as noted in the previous section. To motivate their choice of knapsack function, they also show that Micciancio’s original one-way function is not collision-resistant, nor even universal one-way.
\section{Techniques and ideas}
The techniques used in the paper use the fact that cyclic lattices are closed under cyclic convolution with integer vectors. Furthermore, the lattice points naturally correspond to polynomials in $\mathbb{Z}[\alpha]/(\alpha^n - 1)$.
\\
Convolution is defined as follows. For any $x = (x_0 , . . . , x_{n-1})^T \in \mathbb{R}^n$ , define the rotation of $x$,
denoted as $rot(x)$, to be the vector $(x_{n-1} , x_0 , . . . , x_{n−2})^T$ ; similarly $rot_i(x) = rot(· · · rot(x) · · · )$ is
defined to be the rotation of $x$, taken $i$ times. A lattice $\Latt$ is cyclic if for all $x \in \Latt$, $rot(x) \in \Latt$. For any integer $d \geq 1$, define the rotation matrix $Rot^d (x)$ to be the matrix $[x|rot(x)| · · · |rot^{d-1}(x)]$.
\par
For any ring $R$, the (cyclic) convolution product of $x, y \in R^n$ is the vector $x \otimes y = Rot^n(x) \cdot y$,
with entries 
\begin{align*}
 (x \otimes y)_k = \sum_{i+j = k \text{ mod }n} x_i \cdot y_j
\end{align*}
Hence we observe that in a cyclic lattice $\Latt$, the convolution of any $x \in \Latt$ with any integer vector $y \in \mathbb{Z}_n$ is also in the lattice: $x \otimes y \in \Latt$. This is because all the columns of $Rot^n(x)$ are in $\Latt$, and any integer combination of points in $\Latt$ is also in $\Latt$.
\par
The divisors of $(\alpha^n - 1)$ in $\mathbb{Z}[\alpha]$ correspond to special cyclotomic linear subspaces of $\mathbb{R}^n$ . These
subspaces admit a natural partitioning into complementary pairs of orthogonal subspaces. Even more importantly, the subspaces are closed under cyclic rotation of vector coordinates, and under certain other conditions, these rotations are linearly independent. These facts imply a new connection between the SIVP and SVP problems in cyclic lattices.
\par
The security of the knapsack function comes from using all this structure to impose an algebraic
restriction on the function domain. Looking ahead to the security reduction, this restriction ensures
that collisions in the function are very likely to yield "useful" and short lattice points in a desired
subspace.
\section{Definitions}
We go through some definitions that will help to show the main worst case to average case reduction.
\begin{definition}
	\textit{Cyclotomic Subspace}
	\begin{align*}
	H_\Phi = \{x \in \mathbb{R}^n:\Phi(\alpha)\text{ divides }x(\alpha) \in \mathbb{R}[\alpha]\}.
	\end{align*}
\end{definition}
\begin{definition}
	\textit{SubSIVP} 
	\par
	The cyclotomic (generalized) short independent vectors problem, $SubSIVP_\gamma^\zeta$ , given an $n$-dimensional full-rank cyclic lattice basis $B$ and an integer polynomial $\Phi(\alpha) \neq 0$ mod $(\alpha^n - 1)$ that divides $\alpha^n - 1$, asks for a set of $dim(H_\Phi)$ linearly independent
	(sub)lattice vectors $S \subset \Latt(B) \cap H_\Phi$ such that $||S|| \leq \gamma(n) · \zeta(\Latt(B) \cap H_\Phi )$.
\end{definition}
\begin{definition}
	\textit{SubSVP} 
	\par
	The cyclotomic (generalized) short vectors problem,
	$SubSVP_\gamma^\zeta$ , given an $n$-dimensional full-rank cyclic lattice basis $B$ and an integer polynomial $\Phi(\alpha) \neq 0$ mod $(\alpha^n - 1)$ that divides $\alpha^n - 1$, asks for a (sub)lattice vector $c \in \Latt(B) \cap H_\Phi$ such that $||c|| \leq \gamma(n) · \zeta(\Latt(B) \cap H_\Phi )$.
\end{definition}
\begin{definition}
	\textit{SubIncSVP} 
	\par
	The cyclotomic (generalized) short vectors problem,
	$SubIncSVP_\gamma^\zeta$ , given an $n$-dimensional full-rank cyclic lattice basis $B$ and an integer polynomial $\Phi(\alpha) \neq 0$ mod $(\alpha^n - 1)$ that divides $\alpha^n - 1$, and a nonzero (sub)lattice vector $c \in \Latt(B) \cap H_\Phi$ such that $||c|| > \gamma(n) · \zeta(\Latt(B) \cap H_\Phi )$, asks for a non-zero (sub)lattice vector $c' \in \Latt(B) \cap H_\Phi$ such that $||c'|| \leq ||c||/2$.
\end{definition}
\section{Lemmas}
\begin{lemma}
	$H_\Phi$ is closed under rotation, thats if $c \in H_\Phi$, then $rot(c) \in H_\Phi$
\end{lemma}
\begin{proof}
	{
		\centering
		$rot(c) = \alpha \cdot c(\alpha)$ mod $(\alpha^n - 1)$
		\\
		$(\alpha^n-1) | rot(c) - \alpha \cdot c(\alpha)$
		\\
		$\Phi(\alpha) | rot(c) - \alpha \cdot c(\alpha)$
		\\
		$\Phi(\alpha) | rot(c)$\par
	}
\end{proof}
We omit other proofs and just mention the lemmas and ideas in the paper.
\begin{lemma}
	Let $c \in \mathbb{Z}^n$, and suppose $\Phi(\alpha) \in \mathbb{Z}[\alpha]$ divides $(\alpha^n-1)$ and is coprime to $c(\alpha)$. Then $c, rot(c)....,rot^{deg(\Phi)-1}(c)$ are linearly independent.
\end{lemma}
\begin{lemma}
	Let $a,b \in \mathbb{R}^n$ with $a(\alpha) \cdot b(\alpha) = 0$ mod $(\alpha^n-1)$. Then $\langle a,b \rangle = 0$.
\end{lemma}
\begin{lemma}
	$H_\Phi$ is a linear subspace of $\mathbb{R}^n$ of dimension $n - deg(\Phi)$.
\end{lemma}
\section{Reductions}
\begin{proposition}
	For any $\zeta, \gamma(n)$, there is a deterministic, polynomial-time sublattice-preserving
	reduction from $SubSVP_\gamma^\zeta$ to $SubIncSVP_\gamma^\zeta$.
\end{proposition}
\textit{Informal Proof:} Given an instance of $(B,\Phi(\alpha))$, iteratively reduce the length of $c$ by invoking oracle for $SubIncSVP_\gamma^\zeta$ on $(B,\Phi(\alpha),c)$. If oracle fails we have solved $SubSVP_\gamma^\zeta$. (Note it is easy to show that the iterative process lasts poly number of times).
\begin{proposition}
	For any $\zeta, \gamma(n)$, there is a deterministic, polynomial-time sublattice-preserving
	reduction from $SubSIVP_\gamma^\zeta$ to $SubSVP_\gamma^\zeta$ which makes one oracle call where $\Phi(\alpha) = (\alpha^n - 1)/\Phi_k(\alpha)$ for some $k | n$.
\end{proposition}
\textit{Informal Proof:} We use the $SubSVP$ oracle to find a
short vector in $\Latt(B) \cap H_{\Phi_1}$ , then rotate it to yield $n - 1$ linearly independent vectors, and output that answer for $SSubSIVP$.
\begin{proposition}
	For any $\zeta, \gamma(n)$, there is a deterministic, polynomial-time lattice-preserving
	reduction from $SIVP_{\max(n,2\gamma)}$ on a cyclic lattice of prime dimension to $SubSVP_\gamma^{\lambda_1}$ which makes one oracle call where $\Phi(\alpha) = \Phi_1(\alpha) = \alpha - 1$.
\end{proposition}
\textit{Informal Proof:} The main idea behind the proof is as follows: first, we use the $SubSVP$ oracle to find a
short vector in $\Latt(B) \cap H_{\Phi_1}$ , then rotate it to yield $n - 1$ linearly independent vectors. For the nth
vector, we take the shortest vector in $\Latt(B) \cap H_{\Phi_n}$ , which can be found efficiently; furthermore, it
is an n-approximation to the shortest vector in $L(B) \setminus H_{\Phi_1}.$
\\~\\
$s_i = \sum_{j=1}^n(\mathbf{b}_i)_j = \mathbf{b}_i(1)$ for $i = 1, . . . n$. Because $\alpha - 1$ cannot divide every $b_i(\alpha)$
(otherwise $\Latt(B) \subset H_{\Phi_1}$ , so $\Latt(B)$ would not be full-rank), some $s_i$ must be non-zero.
\\
$\mathbf{s}_i = \mathbf{b}_i \otimes (1,1,.....) = (s_i,s_i....s_i) \in \Latt(B)$. Let $g = gcd(s_1,s_2....s_n)$. Output $\mathbf{g} = (g,g....g)$ as the shortest vector. By the extended Euclidean algorithm, $g$ is an
integer combination of the $s_i$ vectors, hence $g \in \Latt(B)$.
\begin{proposition}
	For any $\gamma(n)$, there is a deterministic, polynomial-time lattice-preserving
	reduction from $SVP_{\max(n,\gamma)}$ on a cyclic lattice of prime dimension to $SubSVP_\gamma^{\lambda_1}$ which makes one oracle call where $\Phi(\alpha) = \Phi_1(\alpha) = \alpha - 1$.
\end{proposition}
\textit{Informal Proof} The idea is same as in the previous reduction where we used the oracle to solve $SIVP$. To use the oracle to solve $SVP$, output minimum of norm of c, and the norm of the $n^{th}$ vector as chosen before.
\\~\\
Hence we have claimed reductions from $SVP_{\max(n,\gamma)}$ to $SubSVP$, and $SubSVP$ to $SubIncSVP$. Thus we now try to show a reduction from $SubIncSVP$ to finding collisions in generalized knapsack with a specific choice of ring. This will show a reduction from $SVP_{\max(n,\gamma)}$ to the main collision resistant hash funcion.
\section{Finding Collisions}
We try to find collisions in the general knapsack function and try to motivate the papers choice of ring $R$ and subset $S$.
\\~\\
\textbf{Generalized Compact Knapsacks}: For any ring R, subset $S \subset R$ and integer $m \geq 1$, the generalized function family $H(R,S,m) = \{f_a : S^m \rightarrow R\}_{a\in R^m}$ is defined by:
\begin{align*}
f_a(x) = \sum_{i=1}^{m}x_i \cdot a_i
\end{align*}
We observe that $f_A$ is linear:
\begin{align*}
f_A(X) + f_A(X') = f_A (X + X')
\end{align*}
For random key $A$, to find a collision with $X'$ it suffices to find a non-zero $X \in S^m$ such that $f_A(X) = 0$, and $||X||_\infty$ is small.(So that the bound on $||X||_\infty$ is still satisfied).
\\~\\
\begin{align*}
f_a(x) = \sum_{i=1}^{m}x_i(\alpha) \cdot a_i(\alpha) mod (\alpha^n - 1)
\end{align*}
We define $X = (x_1,x_2...x_m)$ as follows, for any q that is a divisor of n:
\begin{align*}
x_1(\alpha) = \frac{\alpha^n-1}{\alpha^q-1}
\\
x_j(\alpha) = 0 | j \neq 1
\end{align*}
Now $f_A(X) = 0$ if $a_1(\alpha)$ is divisible by $a^q - 1$, note that this happens with probability $1/p^q$. Over a uniform choice of A, so we have found a specific $X \neq 0$, such that $f_A(X) \neq 0$ with non-negligible probability.
\\~\\
The fact enabling this attack is that $(\alpha^n-1)$ is not irreducible in $\mathbb{Z}_p[\alpha]$. So it easy to find $x(\alpha)$ with small coefficients such that $a(\alpha)x(\alpha) = 0$ mod $(\alpha^n-1)$ and for each divisor of $(\alpha^n-1)$ either $x(\alpha) = 0$ or $a(\alpha) = 0$.
\\
To prevent this we enforce an algebraic constraint on X, informally we require every $x_i(\alpha)$ to be divisible over $\mathbb{Z}[\alpha]$ by $\frac{\alpha^n-1}{\Phi_k(\alpha)}$ for some fixed $k$ dividing $n$. Now essentially the evaluation is performed mod $\Phi_k(\alpha)$. Their choice of subset $S_{D,\Phi}$ is as follows.
\begin{center}
	$S_{D,\Phi} = \{x \in \mathbb{Z}^n_p : ||x||_\infty \leq D$ and $\Phi(\alpha)$ divides $x_\mathbb{Z}(\alpha)$ in $\mathbb{Z}[\alpha]\}$
\end{center}
\section{The main reduction}
We reduce from $SubIncSVP_\gamma^{\eta_\epsilon}$ to collision function $H(\mathbb{Z}_{p(n)} , S_{D(n),\Phi}, m(n))$. Note that this is a worst case to average case reduction on cyclic lattices.
\\
Here we try to show three main properties:
\begin{enumerate}
	\item Sampling an average instance.
	\item Outputting a new vector and showing that it belongs in $\Latt(B) \cap H$.
	\item It has a norm smaller than half the norm of input vector in $SubIncSVP$, with non negligible probability.
\end{enumerate}
\subsection{The algorithm}
\begin{enumerate}
	\item For $i = 1$ to $m$,
	\begin{itemize}
		\item Generate uniform $\mathbf{v}_i \in \Latt(B)\cap H \cap P(Rot^d(\mathbf{c}))$. \cite{Micciancio2002}
		\item Generate noise $\mathbf{y}_i \in H$, according to $D_{H,s}$ for $s = 2||c||/\gamma(n)$. Let $y'_i = y_i$ mod $P(B)$.
		\item Choose $b_i$ so that $Rot^n(\mathbf{c}) \cdot \mathbf{b} = \mathbf{v_i} + \mathbf{y'_i}$ , and let $\mathbf{a_i} = \lfloor \mathbf{b_i} \cdot p \rceil $
	\end{itemize}
	\item Pass A to collision finding oracle, and get collission pairs $X,X'$, let $Z = X - X'$, such that $||Z||_\infty \leq 2D$ and $\Phi(\alpha)$ divides every $z_i(\alpha)$.
	\item Output 
	\begin{align*}
	c' = \sum_{i=1}^m(\mathbf{v_i}+\mathbf{y'_i} - \mathbf{y_i}) \otimes \mathbf{z_i} - \mathbf{c} \otimes \frac{\sum_{i=1}^m \mathbf{a_i} \otimes \mathbf{z_i}}{p}
	\end{align*} 	
\end{enumerate}
\subsection{Correctness}
\begin{itemize}
\item \textit{Property - Instance sampled is average instance}: It is uniform because for choosing b we sample the latter half uniformly from $I^{n-d}$. And the former half is chosen such that ${(I_{d \times d})}^{-1}(v_i + y'_i - w)$, where $w = Rot^n(c) \cdot {(0,0,...(b_i)_d,..,(b_i)_n)}^T$. Since $y'$ is statistically uniform from our choice of c such that the spread in the guassian is more than the smoothing parameter, and $v$ is uniform. We have the former half is uniform, and since the latter half is already sampled uniformly we have that the $b_i's$ are uniform.
\item \textit{Property - Outputting a new vector and showing that it belongs in $\Latt(B) \cap H$}: Convolution of a lattice vector with an integer vector also lies in the lattice, hence the second property holds true. Note that the second term in the output vector is c convoluted with an integer vector due to a convolution z being congruent to zero mod p as we have constructed z from a collision.
\item \textit{Property - Outputted vector is smaller than half of the input vector}: Idea is to use Markov's Inequality and bound on the expected value of the new vector that is outputted.
\end{itemize}
\section{Future Work}
In the same year Lyubashevsky and Micciancio \cite{Lyubashevsky:2006:GCK:2097282.2097300} obtained exceedingly similar results but expressed them in different mathematical language. In particular, by making
many of the same algebraic insights, they constructed collision-resistant hash functions with nearly
identical parameters, based on a worst-case hardness assumption that was similar to the paper studied. They also presented a more general algebraic framework for constructing hash functions, which can be related to problems in algebraic number theory. Due to its generality, their framework may have the potential to admit better constructions, though its current best application essentially matches the collision resistant function considered here.
\par
A more practical instantiation of the function considered here was presented in \cite{Lyubashevsky:2008:SMP:1425852.1425858}. They propose a collection of compression functions that are highly parallelizable and admit very efficient implementations on modern microprocessors. Their constructions were supported with a detailed security analysis of concrete instantiations, and a high-performance software implementation that exploits the inherent parallelism of the $FFT$ algorithm. They claim that the throughput of their implementation is competitive with that of $SHA-256$, with additional parallelism yet to be exploited.

\bibliography{references.bib}{}
\bibliographystyle{plain}

\end{document}
