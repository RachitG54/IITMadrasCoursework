\section{Comparison with known solutions}
	\begin{table}[H]
		\begin{tabular}{|m{2cm} | m{2.1cm} | m{2.3cm} | m{2.6cm} | m{3.1cm} | m{3cm}|}
			\hline
			\textbf{ } & \textbf{Security} & \textbf{Efficiency} & \textbf{Lattice Class} & \textbf{Assumption} & \textbf{Approx. Factor}\\
			\hline
			Ajtai & CRHF & $O(n^2)$ & General & SVP etc. & poly(n) \\ \hline
			Cai, Nerurkar & CRHF & $O(n^2)$ & General & SVP etc. & $n^{4+\epsilon}$ \\ \hline
			Micciancio \vspace{0.3cm} & OWF & $\tilde{O}(n)$ & Cyclic & GDD & $n^{1+\epsilon}$ \\ \hline
			Micciancio, Regev & CRHF & $O(n^2)$ & General & SVP etc. & $\tilde{O}(n)$ \\ \hline
			This \vspace{0.3cm} work& CRHF & $\tilde{O}(n)$ & Cyclic & SVP etc. & $\tilde{O}(n)$ \\
			\hline
		\end{tabular}
		\caption{Comparison of results in lattice-based cryptographic functions with worst-case to average case security reductions, to date. “Efficiency” means the key size and computation time, as a function of the lattice dimension n. “Security” denotes the function’s main cryptographic property.}
	\end{table}
The work studied was similar to Micciancio's work on cylic lattices. However the reduction used to establish collision-resistance differs in many significant ways. Micciancio’s function is proven to be one-way, while the authors is collision-resistant. On the other hand, Micciancio relies on a presumably weaker worst-case assumption than theirs. The stronger assumption, combined with the algebraic
view of cyclic lattices, makes the security reduction tighter and conceptually simpler.
\par
Other works considered are slower in efficiency. The structure of cyclic lattices and the choice of ring admits very efficient implementations of the knapsack function: using a Fast Fourier Transform algorithm. The resulting time complexity of the function is $O(m \cdot n \cdot poly(\log n))$, with key size $O(m \cdot n \log n)$. Where $m$ denotes the number of elements in the key.