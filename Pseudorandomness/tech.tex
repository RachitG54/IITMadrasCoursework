\section{Techniques Overview}
The paper contained two broad techniques. First involved a connection between a certain kind of psudorandom generator and an extractor. The main contribution is the statement of the result, rather than its proof, since it involves a new, more general, way of looking at pseudorandom generator constructions. The Impagliazzo-Wigderson generator \cite{Impagliazzo:1997:PBE:258533.258590} is used in their construction. The analysis of its constructions shows that if the predicate is hard, then it is also computationally hard to distinguish the output of the generator from the uniform distribution. This implication is proved by means of a reduction showing how a circuit that is able to distinguish the output of the generator from the uniform distribution can be transformed into a slightly larger circuit that computes the predicate. The paper mentions a stronger assumption of the predicate being chosen randomly and claims that under this assumption the output is statistically close to uniform. 
\\
Second technique involves converting the Nisan-Wigderson generator \cite{Nisan:1994:HVR:192095.192097} to an extractor. This generator is more simpler and easier to analyse. When we move the proof along similar lines as in the Impagliazzo-Wigderson generator we see that at one point the proof fails as a bound is not satisfied.[See \ref{proof_fail} for further explanation]. To fix this the author uses error correcting codes to achieve a good extractor with better parameters to prior work and the Impagliazzo-Wigderson generator.