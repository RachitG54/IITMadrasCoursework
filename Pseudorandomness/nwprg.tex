\section{Nisan Wigderson PRG to Extractor}
\label{proof_fail} 
The construction in Nisan Wigderson PRG \cite{Nisan:1994:HVR:192095.192097} moves along similar lines(\ref{IWPRG}) and is proved by means of a reduction that shows that if $T$ is a test that distinguishes the output of the generator with predicate $P$ from uniform, then there is a small circuit with one $T$-gate that approximately computes $P$. That is, the circuit computes a predicate that agrees with $P$ on a fraction of inputs noticeably bounded away from $1/2$. As in the previous case we define a bad set of strings where the statistical distance is greater than $\epsilon$. But previously we could say that each circuit should give a different bad string, and hence the size of the strings with the bad set is bounded by the number of possible circuit. But now the result states that the circuit with a fixed size can only approximately evaluate $\langle x \rangle$. Hence for each such $\langle x \rangle$ we can say that there is a circuit of size $S$ that describes a string whose Hamming distance from $\langle x \rangle$ is noticeably less than $1/2$. Since there are about $2^S$ such circuits, the total number of bad strings is at most $2^S$ times the number of strings that can belong to a Hamming sphere of radius about $1/2$. This is the only point where the proof breaks down. Hence the idea here is that we bound the number of circuits in a hamming ball of radius almost half by using error correcting codes. This completes the proof. We assume the following lemma to be true.
\begin{lemma}
(Error Correcting Codes). For every $n$ and $\delta$ there is a polynomial-time computable encoding $EC : \{0,1\}^n \rightarrow \{0, 1\}^{\tilde{n}}$ where $\tilde{n}= poly(n, 1/\delta)$ such that every ball of Hamming radius $(1/2 - \delta)\tilde{n}$ in $\{0, 1\}^{\tilde{n}}$ contains at most $1/{\delta^2}$ codewords. Furthermore $\tilde{n}$ can be assumed to be a power of $2$.
\label{ECcode}
\end{lemma}
Their are many constructions of the error correcting code that achieve the requirements. In particular one can use a Reed-Solomon code concatenated with a Hadamard code. We assume that such a construction is available to us.
\\~\\
We now look at the definition for $(m,l,a)$ design which will be useful for the construction of the generators.
\\~\\
\textit{Design:} For positive integers $m, l, a \leq l,$ and $t \geq l$, $a(m, t, l, a)$
design is a family $S = S_1 , . . . , S_m$ of sets such that
\begin{itemize}
	\item $S_i \subseteq [t]$,
	\item $|S_i| = l$,
	\item $\forall i \neq j \in [m], |S_i \cap S_j| \leq a$.
\end{itemize} 
\begin{lemma}
	\label{design_const}
	For every positive integers $m, l$, and $a \leq l$ there exists a $(m, t, l, a)$ design where $t = e^{\frac{\ln m}{a} + 1} \cdot \frac{l^2}{a}$. Such a design can be computed deterministically in $O(2^t m)$ time.
\end{lemma}
Nisan and Wigderson showed an explicit construction in their paper. \cite{Nisan:1994:HVR:192095.192097}. They showed this construction for $a = \log m$.
\\~\\
We now describe a notation that will be useful in defining the Nisan Wigderson generator. If $S \subseteq [t]$, with $S = \{s_1 , . . . , s_l \}$ (where $s_1 < s_2 < \ldots < s_l$) and $y \in \{0, 1\}^t$ , then we denote by $y_{|S} \in \{0, 1\}^l$ the string $y_{s_1}y_{s_2} \ldots y_{s_l}$. We now define the NW generator.
\begin{definition}
	For a function $f : \{0, 1\}^l \rightarrow \{0, 1\}$ and an $(m, t, l, a)$-design $S = (S_1 , . . . , S_m)$, the Nisan-Wigderson generator $NW_{f,S} : \{0,1\}^t \rightarrow \{0, 1\}^m$ is defined as:
	\begin{align*}
		NW_{f,S}(y) = f(y_{|S_1})\ldots f(y_{|S_m})
	\end{align*}
\end{definition}
The intuition behind the Nisan and Wigderson generator is that if $f$ is hard-on average then $f$ being evaluated at random points will be hard to distinguish by a bounded adversary. In order to reduce the seed, evaluation points are not chosen independently but based on a small corelation generated by the design. Nisan and Wigderson show that such a construction is indeed a PRG.
The main lemma used is as follows:
\begin{lemma}
Let $S$ be an $(m, l, a)$-design, and $T : \{0, 1\}^m \rightarrow \{0, 1\}$. Then there exists a family $\mathbb{G}_T$ (depending on $T$ and $S$) of at most $2^{m2^a +\log m+2}$ functions such that for every function $f : \{0, 1\}^l \rightarrow \{0, 1\}$ satisfying:
\begin{align*}
|Pr_{y \in \{0,1\}^t}[T(NW_{f,S}(y)) = 1] - Pr_{r \in \{0,1\}^m}[T(r) = 1]| \geq \epsilon
\end{align*}
then there exists a function $g : \{0, 1\}^l \rightarrow \{0, 1\}, g \in \mathbb{G}_T$ , such that $g(\cdot)$ approximates $f (\cdot)$ within $1/2 + \epsilon/m$.
\end{lemma}
Nisan and Wigderson proved a similar lemma along these lines and showed that if the function were not a PRG then by using a hybrid argument one could construct a circuit that predicts a bit of the output by seeing its preceding bits. They then use this circuit to contradict the hardness of $f$.
\\~\\
Trevisan showed that this lemma along with the error correcting code is enough to prove that the generator is an extractor. To see this observe that every bad string($\langle x \rangle$) will now have a neighbour in $\mathbb{G}_T$ that is within $1/2 + \epsilon/m$ distance from the encoding of $\langle x \rangle$. Hence the number of bad strings are bounded by the total number of strings in $\mathbb{G}_T$ times the number of codewords that are within the ball of length $1/2 - \epsilon/m$. Carefully choosing the parameters proves that the Nissan Wigderson generator is a $(k,2\epsilon)$ extractor for $t = O(e^{\log (k/2m)} \cdot \frac{l^2}{\log (k/2m)})$.
\\
The final constructor of the extractor function is as follows:
\\~\\
The construction has parameters $n, k \leq n$, $36 \leq m \leq k/2$ and $0 < \epsilon < 2^{-k/12 }$. It can be verified that the constraints on the parameters imply that $2 + 3 \log m + 3 \log (1/\epsilon) < k/2$ (because we have $k/4 > 3 \log 1/\epsilon$ and $k/4 \geq m/2 \geq 2 + 3 \log m$ for $m \geq 36$).
\\
Let $EC : \{0, 1\}^n \rightarrow \{0, 1\}^{\tilde n}$ be as in Lemma \ref{ECcode}, with $\delta = \epsilon/m$, so that $\tilde n = poly(n, 1/\epsilon)$, and define $l = \log \tilde{n} = O(\log n/\epsilon)$.
\\
For an element $u \in \{0, 1\}^n$ , define $\tilde{u} = \langle EC(u) \rangle : \{0, 1\}^l \rightarrow \{0, 1\}$. Let $S = S_1 , . . . , S_m$ be as in Lemma \ref{design_const}, such that
\begin{itemize}
	\item $S_i \subseteq [t]$,
	\item $|S_i| = l$,
	\item $\forall i \neq j \in [m], |S_i \cap S_j| \leq a = \log(k/2m)$
	\item $t = O(e^{\log (k/2m)} \cdot \frac{l^2}{\log (k/2m)})$.
\end{itemize}
Then we observe that from our choice $m > t$ and $Ext : \{0,1\}^n \times \{0,1\}^t \rightarrow \{0,1\}^m$ as
\begin{align*}
	Ext(u,y) = NW_{\tilde{u},S}(y) = \tilde{u}(y_{|S_1}) \ldots \tilde{u}(y_{|S_m}).
\end{align*}