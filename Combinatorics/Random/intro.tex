\section{Introduction}
Erdos and Renyi initiated the theory of random graphs in an earlier paper. In the paper they concentrated on the graphs being connected and the size of the components. In this paper they make an observation that there is a drastic change in the structure of the graph with respect to the number of vertices in the largest component. This change occurs when the number of edges are around half the number of vertices. First they define what it means to be a random graph and the various models that are considered and then the observe the evolution of the random graph as they transition over various phases.