\section{Random Graph}
Erdos and Renyi described what it means be a random graph. They concentrate their study on graphs that are not oriented, without parallel edges and slings. A random graph is defined such that the edges are chosen randomly from all possible edges and that the probability of choossing each edge configuration is same and equal to $\frac{1}{\binom{\binom{n}{2}}{N}}$ for a graph with $n$ vertices and $N$ edges. There is also another view to this random graph process. At time t=1 we can unbiasely choose one out of all the possible $\binom{n}{2}$ edges. At time t=2 we can do the same on remaining edges and so on. The two definitions are equivalent and the second definition interprets the random graph as time. Thus we can look at how this step by step unravelling of the random graphs occur. The study of this process is what they call evolution of a graph.
\\
The main aim of the paper is to show that random graphs exebit very clear cut features. The theorems proved can be categorized into two classes. The theorems of the first class deal with the appearance of certain subgraphs (e.g. trees, cycles of a given order etc.) or components, or other local structural properties, and show that for many types of local structural properties a definite threshold $A(n)$ can be given. The theorems of the second class are of similar type, only the properties considered are not of a local character, but global properties of the graph such as connectivity, total number of components, etc.