\section{Remarks}
This paper presented many applications in various fields of mathematics. The proofs in the paper involved the method of contradiction by construction of clever polynomials that algebraically represented the problem. There were a few more ways in which the main theorem \ref{theorem} was modified such as in the permanent lemma where we used the inequality to achieve an equality. 
\\
Most proofs presented in the paper algebraic and hence non-constructive in the sense that they supply no efficient algorithm for solving the corresponding algorithmic problems. An interesting extension to this paper would be to find an efficient algorithm to find a point $(s_1,s_2, \ldots, s_n)$ such that the polynomial evaluates to non zero value in theorem \ref{theorem}.