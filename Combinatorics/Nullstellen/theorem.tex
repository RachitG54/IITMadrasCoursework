\section{Main Theorem}
Combinatorial nullstellensatz is based on the observation of Hilbert's nullstellensatz which states that if $F$ is an algebraically closed field, and $f$, $g_1$ , $\ldots$ , $g_m$ are polynomials in the ring of polynomials $F[x_1 ,\ldots , x_n]$, where $f$ vanishes over all common zeros of $g_1 ,\ldots, g_m$ , then there is an integer $k$ and polynomials $h_1 , \ldots, h_m$ in $F[x_1 , \ldots , x_n]$ so that $f^k = \sum_{i=1}^{n}h_ig_i$.
\\
This observation gives rise to the following theorem:
\begin{theorem}
	Let $F$ be an arbitrary field, and let $f = f(x_1 ,\ldots, x_n)$ be a polynomial in $F[x_1 ,\ldots, x_n]$. Suppose the degree $deg(f)$ of $f$ is coefficient of $\sum_{i=1}^n t_i$, where each $t_i$ is a nonnegative integer, and suppose the coefficient of $\Pi_{i=1}^n {x_i}^{t_i}$ in $f$ is nonzero. Then, if $S_1 ,\ldots, S_n$ are subsets of $F$ with $|S_i| > t_i$ , there are $s_1 \in S_1 , s_2 \in S_2 ,\ldots, s_n \in S_n$ so that $f(s_1 , \ldots , s_n ) \neq 0$.
	\label{theorem}
\end{theorem}