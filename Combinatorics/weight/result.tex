\section{Main Result}
The main result of the paper is as follows:
\begin{theorem}
	Let $G = (V, E)$ be a simple graph with $mind(G) \leq 2$. Let $U$ be a nonempty subset of $V(G)$. Let $F$ be a graph obtained by adding two new vertices $u, v$ to $G$ and joining them to each vertex of $U$. Let $H$ be a graph obtained from $F$ by joining $u$ and $v$ by an edge. Then $mind(F) \leq 2$ and $mind(H) \leq 2$.
\end{theorem}
The theorem allows for recursive constructions of many graphs with low monomial index.
\\
From the main theorem the following observation follows: if $G \neq K_2$ is a clique, complete bipartite graph, or a tree, then $mind(G) \leq 2$. This proves that these classes of graphs are three weight colorable. 
\\
Other properties that follow are: 
\\
Every graph $G$ is an induced subgraph of a graph $H$ such that $mind(H) \leq 2$ and $\chi(H) \leq \chi(G) + 1$.
\\
Every graph $G$ can be transformed into a graph $H$ with $mind(H) \leq 2$ by subdividing each edge of $G$ with at most three vertices. Every graph $G$ can be transformed into a graph $H$ with $mind(H) \leq 2$ by subdividing each edge of $G$ with at most three vertices.
\\
These results bring the hope that the conjecture that every graph except $K_2$ is three weight colorable. The main question that still remains open is whether there is a finite bound on the monomial index for graphs.