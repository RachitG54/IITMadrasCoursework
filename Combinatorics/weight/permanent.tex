\section{Permanent}
Combinatorial nullstellensatz implies that if $mind(G) \leq k$ then $G$ is $(k + 1)$-weight choosable. Hence to prove that certain class of graphs are $3$ weight colorable, we need to show that $mind(G) \leq 2$. To study the monomials in the expansion of $P_G$ we look at the permanent of matrices.
\\
The permanent rank of a matrix $A$ is the size of a largest square submatrix with non-zero permanent. Let $A^{(k)} = [A, \ldots , A]$ be a matrix formed of $k$ copies of a matrix $A$. $pind(A)$ is the minimum $k$ for which $A(k)$ has the permanent rank equal to the size of $A$.
\begin{lemma}
	Let $A = (a_{ij})$ be a square matrix of size $m$ and finite permanent index. Let $P(x_1 , \ldots , x_m)=\prod_{i=1}^{m}(a_{i1} x_1 + \ldots + a_{im} x_m)$. Then $mind(P) = pind(A)$. 
\end{lemma}
Hence we can use $pind(A)$ to estimate $mind(P)$ and hence the weight color.
\begin{lemma}
	Let $A$ and $L$ be square matrices of size $n$ such that each column of $L$ is a linear combination of columns of $A$. Let $n_j$ be the number of those columns of $L$ in which the $j^{th}$ column of $A$ appears with a non-zero coefficient. If $n_j \leq r$ and $per(L) \neq 0$, then $pind(A) \leq r$.
\end{lemma}