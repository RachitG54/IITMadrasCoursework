\section{Using the Polynomial Method}
We first define the appropriate terms and lemmas that will be useful in the proof.
\begin{definition}
	($(\delta, \gamma)$-Kakeya Set). A set $K \subset \F^n$ is a $(\delta,\gamma)$-Kakeya Set if there exists a set $L \subset \F^n$ of size at least $\delta \cdot q^n$ such that for every $x \in L$ there is a line in direction $x$ that intersects $K$ in at least $\gamma \cdot q$ points.
\end{definition}
We mention the schwartz zippel lemma that bounds the number of roots of a n variate degree d polynomial.
\begin{lemma}
	(Schwartz-Zippel) Let $f \in \F[x_1 , \ldots , x_n]$ be a non zero polynomial with $deg(f) \leq d$. Then
	\begin{align*}
		|\{x \in \F^n | f(x) = 0 \}| \leq d \cdot q^{n-1}
	\end{align*}
\end{lemma}
We now state the theorem:
\begin{theorem}
	Let $K \subset \F^n$ be a $(\delta, \gamma)$-Kakeya Set. Then
	\begin{align*}
		|K| \geq \binom{d+n-1}{n-1}
	\end{align*}
	where
	\begin{align*}
		d = \lfloor q \cdot \min{\{\delta,\gamma\}} \rfloor - 2
	\end{align*}
\end{theorem}
Suppose that the theorem is false then the number of monomials of degree $d$ is larger than the size $K$. Therefore, there exists a homogenous $n$ variate degree $d$ polynomial $g$ such that $g$ is not the zero polynomial and $\forall x \in K, g(x) = 0$. Note that this is a consequence of solving a system of linear equations. We now show that $g$ has too many zeros and hence must be identically zero.
\\
\begin{comment}
Consider the set
\begin{align*}
	K' = \{c \cdot x | x \in K, c \in \F \}
\end{align*}
Since $g$ is zero on $K$ and it is a homogenous polynomial we have that $g$ is zero on $K'$.
\\
\end{comment}
Since $K$ is a $(\delta, \gamma)$-Kakeya set, there exists a set $L \subset \F^n$ of size at least $\delta ·\cdot q^n$ such that for every $y \in L$ there exists a line with direction $y$ that intersects $K$ in at least $\gamma \cdot q$ points. We now claim that $g$ is zero on $L$ as well. The main idea behind this claim is that if a degree $d$ univariate polynomial has $(d+1)$ zeros then it must be identically zero on the entire space.
\\
Since size of $L$ is $\delta \cdot q^n$, we have that a $\delta$ fraction of points are roots. But as $\delta > d/q$ we have a contradiction from the schwartz zippel lemma.
\\
The paper then presented a similar extension to show that $|K| \geq \binom{q+n-1}{n}$. This gives an answer to the question answered by Wolff. 