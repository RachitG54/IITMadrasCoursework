\section{Introduction}
A Kakeya set is a subset of $\F^n$ , where $\F$ is a finite field of $q$ elements, that contains a line in every direction.
\\
This problem follows from the study of the Kakeya needle problem(1917). The problem is as follows: What is the least area in the plane required to continuously rotate a needle of unit length and zero thickness around completely (i.e. by $360^\circ$)? Besicovitch in 1919 showed that there exists Kakeya sets ${\R}^2$ of arbitrarily small area. This paper studies the version in the setting of finite fields. It answers a question from Wolff who in 1999 stated the conjecture that the size of every Kakeya set is at least $C_n \cdot q^n$ , where $C_n$ depends only on $n$. The lower bound showed by Wolff was of the form $C_n \cdot q^{(n+2)/2}$ . This bound was further improved both for general n and for specific small values of n (e.g for n = 3, 4). For general n, the previous best lower bound was $C_n \cdot q^{4n/7}$. Dvir proved the conjecture by using the polynmial method. We briefly summarize and comment on his proof.