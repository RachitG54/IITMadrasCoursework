\section{Applications}
\subsection{Theorem of Chevalley and Warning}
\begin{theorem}
	Let $p$ be a prime, and let $P_1 = P_1(x_1 , \ldots , x_n), \ldots , P_m = P_m(x_1 , \ldots , x_n)$ be $m$ polynomials in the ring $Z_p[x_1 , \ldots , x_n]$. If $n > \sum_{i=1}^{m}deg(P_i)$ and the polynomials $P_i$ have a common zero $(c_1 , \ldots , c_n)$, then they have another common zero.
\end{theorem}
This theorem was proved by Chevalley in 1935. The interesting observation we make in the proof of this theorem is that the main theorem \ref{theorem}, shows that there exists a solution where the polynomial is non zero. To prove we construct a polynomial which must be zero over a fixed space if our assumption is false and then use combinatorial nullstellensatz to arrive at a contradiction.
\\
This method of contradiction is primarily the main idea which is applied to most proofs in the paper.
\subsection{Additive number theory}
\begin{theorem}
	If $p$ is a prime, and $A$, $B$ are two nonempty subsets of $Z_p$ , then 
	\\
	$|A + B| \geq min \{p, |A| + |B| - 1 \}$.
\end{theorem}
This result is called the Cauchy-Davenport Theorem, which has numerous applications in Additive Number Theory. Here again we use the same idea of coming up with a construction of a polynomial that has many roots if the assumption is false. This theorem gave a new proof to a lemma of Lagrange in his well known 1770 paper that shows that any integer is a sum of four squares.
\subsection{Restricted Sums}
We define operation $\oplus$ on sets for a $k$ variate polynomial $h$ as follows:
\begin{align*}
	\oplus_h \sum_{i=0}^{k}A_i = \{a_0 + \ldots + a_k : a_i \in A_i, h(a_0,\ldots,a_k) \neq 0\}.
\end{align*}
\begin{theorem}
	Let $p$ be a prime and let $h = h(x_0 , \ldots , x_k)$ be a polynomial over $Z_p$ . Let $A_0, A_1 , \ldots , A_k$ be nonempty subsets of $Z_p$ , where $|A_i| = c_i + 1$ and define $m = \sum_{i=0}^{k}c_i - deg(h)$. If the coefficient of $\prod_{i=0}^{k}x_i^{c_i}$ in
	\begin{align*}
		{(x_0+\ldots+x_k)}^mh(x_0 , \ldots , x_k)
	\end{align*}
	is non zero mod $p$ then,
	\begin{align*}
		|\oplus_h \sum_{i=0}^{k}A_i | \geq m+1
	\end{align*}
\end{theorem}
There are plenty of applications of this theorem in the paper. The proof of the theorem is consistent with the idea of using contradiction on a construction of a polynomial.

\subsection{Set additions in vector spaces over prime fields}
A triple $(r, s, n)$ of positive integers satisfies the Hopf-Stiefel condition with respect to a prime $p$ if
\begin{center} 
	$\binom{n}{k}$ is divisible by $p$ for every integer $k$ satisfying $n-r < k < s$.
\end{center}
This condition is useful in Topology.
\\
Let $\beta_p(r, s)$ denote the smallest integer $n$ for which the triple $(r, s, n)$ satisfies Hopf-Stiefel condition with respect to a prime $p$.
\begin{theorem}
	If $A$ and $B$ are two finite nonempty subsets of a vector space $V$ over $GF(p)$, and $|A| = r$, $|B| = s$, then $|A + B| \geq \beta_p(r, s)$.
\end{theorem}
In fact this result is sharp for all $r$ and $s$.


\subsection{Graphs and Subgraphs}
\begin{theorem}
	For any prime $p$, any loopless graph $G = (V, E)$ with average degree bigger than $2p - 2$ and maximum degree at most $2p - 1$ contains a $p$-regular subgraph.
\end{theorem}
The proof of this theorem follows a different idea than the previous proofs(earlier proofs used the method of contradiction), here we construct a polynomial that if in a fixed space it has a non zero evaluation then it can be shown to be of a $p$-regular subgraph. This shows the varied applicability of this tool. One more interesting non trivial application along these lines is the following theorem:
\begin{theorem}
	Let $p$ be a prime, and let $G = (V, E)$ be a graph on a set of $|V| > d(p - 1)$ vertices. Then there is a nonempty subset $U$ of vertices of $G$ such that the number of cliques of $d$ vertices of $G$ that intersect $U$ is $0$ modulo $p$.
\end{theorem}
Some versions of these results arise in the study of the minimum possible degree of a polynomial that represents the OR function of n variables.
\subsection{Graph Coloring}
The technique is applied to study the $\textit{choosability}$ properties of a graph. By choosability we mean the following:
\\
If $G = (V, E)$ is a (finite, directed or undirected) graph, and $f$ is a function that assigns to each vertex $v$ of $G$ a positive integer $f(v)$, we say that $G$ is $f$-choosable if, for every assignment of sets of integers $S(v) \subset Z$ to all the vertices $v \in V$ , where $|S(v)| = f(v)$ for all $v$, there is a proper vertex coloring $c : V \rightarrow Z$ so that $c(v) \in S(v)$ for all $v \in V$ . The graph $G$ is $k$-choosable if it is $f$-choosable for the constant function $f(v) \equiv k$. The choice number of $G$, denoted $ch(G)$, is the minimum integer $k$ so that $G$ is $k$-choosable. Obviously, this number is at least the classical chromatic number $\chi(G)$ of $G$. The choice number of the line graph of $G$, which we denote here by $ch'(G)$, is usually called the list chromatic index of $G$, and it is clearly at least the chromatic index $\chi'(G)$ of $G$.
\begin{theorem}
	Let $G$ be a graph on $3n$ vertices, whose set of edges is the disjoint union of a Hamilton cycle and $n$ pairwise vertex-disjoint triangles. Then the choice number and the chromatic number of $G$ are both $3$.
\end{theorem}
\begin{theorem}
	The choice number of every planar bipartite graph is at most $3$.
\end{theorem}
The above the theorem along with the Four Color Theorem gives the following result:
\begin{theorem}
	For every 2-connected cubic planar graph G, ch'(G) = 3
\end{theorem}
It is possible to extend this proof to any d-regular planar multigraph with chromatic index $d$.
\subsection{Permanent Lemma}
\begin{lemma}
	Let $A = (a_{ij})$ be an $n$ by $n$ matrix over a field $\F$ , and suppose its permanent $Per(A)$ is nonzero (over $\F$). Then for any vector $b = (b_1 , b_2 , \ldots, b_n) \in \F_n$ and for any family of sets $S_1 , S_2 , \ldots , S_n$ of $\F$ , each of cardinality $2$, there is a vector $x \in S_1 \times S_2 \times \ldots \times S_n$ such that for every $i$ the $i^{th}$ coordinate of $Ax$ differs from $b_i$.
\end{lemma}
The above lemma follows from the main theorem \ref{theorem}. This lemma can be extended to interesting applications where we observe how it can be applied to give equalities rather than inequalities. This technique can be seen in the following theorem:
\begin{theorem}
	For any prime $p$, any sequence of $2p - 1$ members of $Z_p$ contains a subsequence of cardinality $p$ the sum of whose members is $0$(in $Z_p$).
\end{theorem}
\subsection{Ideals of polynomials and combinatorial properties}
\begin{theorem}
	(Li and Li) A graph $G$ does not contain an independent set of $k + 1$ vertices if and only if the graph polynomial $f_G$ lies in the ideal generated by all graph polynomials of unions of $k$ pairwise vertex disjoint complete graphs that span its set of vertices.
\end{theorem}
\begin{theorem}
	(Kleitman and Lovász) A graph $G$ is not $k$ colorable if and only if the graph polynomial $f_G$ lies in the ideal generated by all graph polynomials of complete graphs on $k + 1$ vertices.
\end{theorem}
\begin{theorem}
	(Alon and Tarsi) A graph $G$ on the $n$ vertices $\{1, 2, \ldots, n\}$ is not $k$ colorable if and only if the graph polynomial $f_G$ lies in the ideal generated by the polynomials ${x_i}^k-1$, ($1 \leq i \leq n$).
\end{theorem}
\begin{theorem}
	The $3$-uniform hypergraph $H = (V, E)$ is not $2$-colorable if and only if the polynomial
	\begin{align*}
		\prod_{e \in E}[(\sum_{v \in e}x_v)^2 - 9]
	\end{align*}
	lies in the ideal generated by the polynomials $\{ {x_v}^2 − 1 : v \in V \}$.
\end{theorem}